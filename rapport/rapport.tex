\documentclass[12pt,a4paper,openany]{book}

\usepackage{lmodern}
\usepackage[svgnames]{xcolor} % Required to specify font color
\input{/home/aroquemaurel/cours/includesLaTeX/couleurs.tex}

\usepackage[utf8]{inputenc}
\usepackage[T1]{fontenc}
\usepackage[francais]{babel}
\usepackage[top=1.7cm, bottom=1.7cm, left=1.7cm, right=1.7cm]{geometry}
\usepackage{verbatim}
\usepackage[urlbordercolor={1 1 1}, linkbordercolor={1 1 1}, linkcolor=vert1, urlcolor=bleu, colorlinks=true]{hyperref}
\usepackage{tikz} %Vectoriel
\usepackage{listings}
\usepackage{fancyhdr}
\usepackage{multido}
\usepackage{amssymb}
\usepackage{float}
\usepackage{graphicx} % Required for box manipulation

\newcommand{\titre}{Programmation événementielle}
\newcommand{\subtitle}{DM \no{}1}
\newcommand{\auteur}{Antoine de \bsc{Roquemaurel}}
\newcommand{\formation}{L3 Informatique}
\newcommand{\semestre}{6}
\newcommand{\annee}{2014}
\newcommand{\prof}{}


\newcommand{\pole}{}
\newcommand{\sigle}{pge}
\input{/home/aroquemaurel/cours/includesLaTeX/listings.tex} %prise en charge du langage algo
\input{/home/aroquemaurel/cours/includesLaTeX/l2/cours.tex}
\input{/home/aroquemaurel/cours/includesLaTeX/remarquesExempleAttention.tex}
\input{/home/aroquemaurel/cours/includesLaTeX/polices.tex}
\input{/home/aroquemaurel/cours/includesLaTeX/affichageChapitre.tex}
\input{/home/aroquemaurel/cours/includesLaTeX/couverture.tex}
\makeatother

\begin{document}
	\thispagestyle{empty} % Removes page numbers
	\titleBC 
	\setcounter{tocdepth}{2}
	\setcounter{secnumdepth}{3}
	\chapter*{Avant-propos}
	Ce rapport contient les choix d'implémentation et de conception pour un projet dans le cadre de la L3 Informatique parcours Ingénierie des Systèmes
	Informatiques de l'université Toulouse III -- Paul Sabatier.

	Il s'agit d'un logiciel permettant la gestion des bulletins d'un établissement scolaire.

	Celui-ci doit être développé en Java en utilisant la bibliothèque graphique \texttt{Swing}, ceci à l'aide de l'éditeur WYSIWYG Matisse présent dans
	\textit{Netbeans}. Ci-joint à ce rapport est donc présent les sources du logiciel sous forme de projet Netbeans.

	\tableofcontents
	\chapter{Conception}
		% TODO diagramme de classe
		\section{Les données métiers}
		\section{L'interface Homme Machine}
			\subsection{Les formulaires}
			\subsection{Les modèles}
			\subsection{Les rendus}
	\chapter{Choix d'interface}
	\chapter{Données de tests}
	Afin de pouvoir tester le logiciel dans de bonnes conditions, des données de tests sont insérés au lancement du logiciel. Pour cela, il faut
	exécuter la méthode \texttt{main} présent dans le package \texttt{managementStudent}. Celle-ci va insérer automatiquement un proviseur, des
	professeurs, des classes avec leurs élèves associés ainsi que des notes : Toutes les données de lectures peuvent être essayés, il est ensuite
	possible d'ajouter et d'éditer des données depuis le logiciel.

	\begin{remarque}
		Il n'était demandé aucune sérialisation ou stockage des données dans une base de donées, ainsi une fois le logiciel fermée, toutes les données
		sont définitivement perdues. Le lancement du logiciel réinsèrera le jeu d'essais.

		Il aurait été possible de stocker les données à l'aide d'une base \textit{Sqlite} afin d'avoir la puissance et la souplesse des bases de
		données, ou une simple sérialisation de nos objets aurait pu faire l'affaire: ces deux solutions ont l'avantage d'être rapide à mettre en place.
	\end{remarque}
\end{document}


